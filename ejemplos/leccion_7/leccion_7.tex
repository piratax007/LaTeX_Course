\documentclass[10pt,a4paper]{article}
\usepackage[utf8]{inputenc}
\usepackage[spanish]{babel}
\usepackage[table]{xcolor}
\usepackage{amsmath}
\usepackage{amsfonts}
\usepackage{amssymb}
\usepackage{lipsum}
\usepackage{vmargin}
\setmargins{2.5cm} % margen izquierdo
{1.5cm} % Margen superior
{16.5cm} % Ancho del área de impresión
{23.42cm} % Largo del área de impresión
{10pt} % Espacio de encabezado
{1cm} % Espacio entre encabezado y texto
{0pt} % Espacio de pie de página
{1cm} % Espacio entre el texto y el pie de pagina
\usepackage{hyperref}
\hypersetup{
colorlinks = true,
linkcolor = blueSpider,
filecolor = ochre,
urlcolor = myBlue,
linktoc = page, % En la tabla de contenido, asigna los enlaces a los número de las páginas no a los títulos
citecolor = ochre,
}
\usepackage{natbib}


\usepackage{xkeyval}
\usepackage{minted}
\usepackage{tikz}

\newcommand{\bftt}[1]{\textbf{\texttt{#1}}}
\newcommand*\keystroke[1]{%
  \tikz[baseline=(key.base)]
    \node[%
      draw,
      fill=white,
      drop shadow={shadow xshift=0.25ex,shadow yshift=-0.25ex,fill=black,opacity=0.75},
      rectangle,
      rounded corners=2pt,
      inner sep=1pt,
      line width=0.5pt,
      font=\scriptsize\sffamily
    ](key) {#1\strut}
  ;
}
\newcommand{\keystrokebftt}[1]{\keystroke{\bftt{#1}}}
\newcommand{\comment}[1]{{\color[HTML]{008080}\textit{\textbf{\texttt{#1}}}}}
\newcommand{\cmd}[1]{{\color[HTML]{008000}\bftt{#1}}} % comandos
\newcommand{\bs}{\char`\\}
\newcommand{\cmdbs}[1]{\cmd{\bs#1}}
\newcommand{\lcb}{\char '173}
\newcommand{\rcb}{\char '175}
\newcommand{\cmdbegin}[1]{\cmdbs{begin\lcb}\bftt{#1}\cmd{\rcb}}
\newcommand{\cmdend}[1]{\cmdbs{end\lcb}\bftt{#1}\cmd{\rcb}}
\newcommand{\wllogo}{\textbf{Overleaf}}

\newenvironment{exampletwouptiny}
  {\VerbatimEnvironment
   \begin{VerbatimOut}{example.out}}
  {\end{VerbatimOut}
   \setlength{\parindent}{0pt}
   \fbox{\begin{tabular}{l|l}
   \begin{minipage}{0.55\linewidth}
     \inputminted[fontsize=\scriptsize,resetmargins]{latex}{example.out}
   \end{minipage} &
   \begin{minipage}{0.35\linewidth}
     \setlength{\parskip}{6pt plus 1pt minus 1pt}%
     \raggedright\scriptsize\input{example.out}
   \end{minipage}
   \end{tabular}}}
   
\newenvironment{exampletwouptinynoframe}
  {\VerbatimEnvironment
   \begin{VerbatimOut}{example.out}}
  {\end{VerbatimOut}
   \setlength{\parindent}{0pt}
   \begin{tabular}{l|l}
   \begin{minipage}{0.55\linewidth}
     \inputminted[fontsize=\scriptsize,resetmargins]{latex}{example.out}
   \end{minipage} &
   \begin{minipage}{0.35\linewidth}
     \setlength{\parskip}{6pt plus 1pt minus 1pt}%
     \raggedright\scriptsize\input{example.out}
   \end{minipage}
   \end{tabular}}

\newcommand\diag[4]{%
  \multicolumn{1}{p{#2}|}{\hskip-\tabcolsep
  $\vcenter{\begin{tikzpicture}[baseline=0,anchor=south west,inner sep=#1]
  \path[use as bounding box] (0,0) rectangle (#2+2\tabcolsep,\baselineskip);
  \node[minimum width={#2+2\tabcolsep-\pgflinewidth},
        minimum  height=\baselineskip+\extrarowheight-\pgflinewidth] (box) {};
  \draw[line cap=round] (box.north west) -- (box.south east);
  \node[anchor=south west] at (box.south west) {#3};
  \node[anchor=north east] at (box.north east) {#4};
 \end{tikzpicture}}$\hskip-\tabcolsep}}
\usepackage{xcolor}

\definecolor{blue254}{HTML}{25289E}
\definecolor{orange22}{HTML}{E55500}
\definecolor{myBlue}{HTML}{027FDF}
\definecolor{negative}{HTML}{181818}
\definecolor{positive}{HTML}{AA3939}


\title{Ejemplos Lección 7 - Referencias bibliográficas}
\author{Mg. Fausto Mauricio Lagos Suárez}
\date{\today}

\begin{document}
\maketitle
\renewcommand{\refname}{Bibliografía} % \bibname si se trata de la clase de documento book
\renewcommand{\contentsname}{Tabla de Contenido}

\tableofcontents

\section{Construir la base de datos bibliográfica}

Lo primero para manejar referencias bibliográficas en un documento \LaTeX{} es disponer de una base de datos bibliográfica la cual deberá estar almacenada en un documento de texto plano con extensión \bftt{.bib}, estas bases de datos, al ser texto plano, pueden ser creadas con cualquier editor de texto plano como \bftt{emacs} (en GNU/Linux). Cada entrada de un archivo \bftt{.bib} tiene unos campos específicos a completar (algunos obligatorio, otros opcionales) de acuerdo al tipo de documento al que se refiera, estos son algunos ejemplos:

\begin{minted}[frame=single]{latex}
@BOOK{Carnevale2006,
author = {Carnevale, N.T. and Hines, M.L.},
title = {The Neuron Book},
publisher = {Cambridge University Press},
year = {2006}
}

@ARTICLE{Barden_Zumino, 
author = {Barden, William and Zumino, Bruno},
title = {Consistent and covariant anomalies in gauge and gravitational theories},
year = {1984},
journal = {Nuclear-Phys, B},
volume = {224},
number = {2},
pages = {421--453}
}
\end{minted}

El primer elemento necesario es la clave bibliográfica, es decir el nombre que se le da al documento dentro de la lista bibliográfica, es el nombre utilizado para referirse a este documento en particular mediante el comando \cmdbs{cite}, los demás elementos dependen del tipo de documento en particular y no todos son obligatorios.

A pesar de que se puede crear una base de datos \bftt{.bib} completamente desde cero agregando entrada por entrada en un archivo de texto plano, esta es una opción poco eficiente y que puede estar acompañada de errores en la edición de las entradas, errores que resultaran en problemas al momento de compilar el documento.

Existen alternativas a la edición manual del archivo \bftt{.tex}, algunas de ellas disponibles en la nube, tales como \href{https://www.zotero.org/}{Zotero} o \href{https://www.mendeley.com/}{Mendeley}, las cuales cuentan con aplicación \emph{Desktop} disponible para su descarga gratuitamente (Zotero bajo licencia de software libre AGPL) y multiplataforma; también existen excelentes opciones fuera de la nube (sin servidores de sincronización) tales como \href{http://www.jabref.org/}{jabref} (Software Libre y multiplataforma) que permite extraer la información bibliográfica directamente desde sitios tales como \href{https://books.google.com/}{Google Books}, \href{https://scholar.google.com/}{Google Académico} o \href{https://arxiv.org/}{arXiv} o archivos \bftt{bibtex}.

\section{Agregar la base de datos bibliográfica al documento}

Se deben agregar los comando \cmdbs{bibliographystyle} y \cmdbs{bibliography} en el lugar donde se quiera aparezca la lista bibliográfica. 

\begin{minted}[frame=single]{latex}
\bibliographystyle{amsplain}
\bibliography{archivo_bib-1,archivo_bib-n}
\end{minted}

Existen muy variados \emph{estilos} de presentación de las referencias bibliográficas, algunos de los cuales requieren cargar algún paquete específico en el preámbulo que puede ofrecer alguna extensión a los comandos de citación generales, los estilos más comunes son \bftt{amsplain} un estilo de presentación de la referencia mediante un número de indexado en la lista bibliográfica, \bftt{amsalpha} un estilo que presenta la referencia bibliográfica mediante un ``acrónimo'' que es combinación del apellido o nombre el autor y el año de publicación, \bftt{apalike} un estilo \bftt{autor - año} tipo normas APA y el estilo \bftt{agsm} (correspondiente al estilo Harvard) del paquete \bftt{natbib}.

\subsection{Citas bibliográficas}

En un documento pueden incluirse citas bibliográficas o no, en el caso en el que no se incluyan citas bibliográficas pero se quiera agregar una lista bibliográfica el comando \cmdbs{nocite\{*\}} se encargará de incluir toda la lista bibliográfica sin importar si se ha citado o no, en el caso en el que se quieran agregar a lista elementos específicos que no se han citado en el documento puede usarse \cmdbs{nocite\{clave\}} donde \bftt{clave} es la clave bibliográfica del documento a incluir; en cualquier otro caso, en el que se quiera agregar citas bibliográficas dentro del documento, el comando \cmdbs{cite\{clave\}} se encargar de crear la cita en el \emph{estilo} elegido y de generar la referencia cruzada correspondiente, el paquete \cmd{hyperref} es el encargado de generar el enlace dinámico.

\subsubsection{Ejemplos de citación bibliográfica}

\lipsum[2]

Aquí aparecerá una cita bibliográfica a \cite{Bellman2013} 

\subsubsection{Compilación con referencias bibliográficas}

La compilación de documentos \LaTeX{} que contienen referencias bibliográficas depende del editor que este utilizando, en particular, en el caso de \bftt{Texmaker} primero debe ejecutarse el programa \bftt{bibtex} con \keystroke{F11} y luego compilar el documento dos veces con \keystroke{F1}, en el caso de \wllogo{} la compilación es automática.

Algunas veces, cuando se prueban diferentes estilos bibliográficos en el mismo documento se generan errores de compilación al cambiar de uno a otro, esto sucede porque las referencias bibliográficas citadas se cargan en el archivo \bftt{.aux} con los campos necesarios para un estilo específico, al cambiar el estilo puede que la información contenida en el archivo \bftt{.aux} no se ajuste al nuevo estilo seleccionado, en una instalación local, por ejemplo con \bftt{texmaker} es recomendable borrar el archivo \bftt{.aux} antes de ejecutar el programa \bftt{bibtex} y compilar el documento con un nuevo estilo bibliográfico, en el caso de \wllogo{} lo recomendable es utilizar la opción \bftt{compile from scrath}.

\lipsum[1]

\phantomsection
\addcontentsline{toc}{section}{Bibliografía} % Incluye la bibligrafía en la tabla de contenido
\bibliographystyle{agsm}
\bibliography{biblio}
\nocite{Campbell2011}

\end{document}