\documentclass[10pt,a4paper]{article}
\usepackage[utf8]{inputenc}
\usepackage[spanish]{babel}
\usepackage{amsmath}
\usepackage{amsfonts}
\usepackage{amssymb}
\usepackage{paracol}
\usepackage{subcaption}

\title{Tomando del control de \LaTeX}
\author{Mg. Fausto M. Lagos S.}
\date{\today}

\usepackage{xcolor}

\definecolor{blue254}{HTML}{25289E}
\definecolor{orange22}{HTML}{E55500}
\definecolor{myBlue}{HTML}{027FDF}
\definecolor{negative}{HTML}{181818}
\definecolor{positive}{HTML}{AA3939}
\usepackage{xstring} % Paquete necesario para cargar condicionales en newcommand

\newcommand{\Fourier}[1][eq]{
	\IfEqCase {#1}{
		{eq}{
			\begin{equation}
				\hat{f}(\xi) = \int_{-\infty}^{\infty}f(x)e^{-2\pi ix\xi}dx
			\end{equation}
		}
		{disp}{
			\[
				\hat{f}(\xi) = \int_{-\infty}^{\infty}f(x)e^{-2\pi ix\xi}dx
			\]
		}
		{inLine}{
			$\hat{f}(\xi) = \int_{-\infty}^{\infty}f(x)e^{-2\pi ix\xi}dx$
		}
	}
}

\newcommand{\Wave}{
\begin{equation}
	\frac{\partial ^2u}{\partial t^2} = c^2\frac{\partial ^2u}{\partial x^2}
\end{equation}
}

\newcommand{\Maxwell}{
	\begin{align*}
		\nabla \cdot E &= 0 & \nabla \times E &= -\frac{1}{c}\frac{\partial H}{\partial t} \\
		\vspace{5pt} \\
		\nabla \cdot H &= 0 & \nabla \times H &= \frac{1}{c}\frac{\partial E}{\partial t}
	\end{align*}
}

\newcommand{\Schrodinger}{
	\begin{equation}
		i\hbar \frac{\partial}{\partial t}\psi = \hat{H}\psi
	\end{equation}
}
%% -------------------------------------------------------------------------
%% 2017 por Fausto M. Lagos S. <piratax007@protonmail.ch>
%% 
%% Este trabajo puede ser distribuido o modificado bajo los
%% términos y condiciones de la LaTeX Project Public License (LPPL) v1.3C, 
%% o cualquier versión posterior. La última versión de esta licencia
%% puede verse en:
%% http://www.latex-project.org/lppl.txt
%% 
%% -------------------------------------------------------------------------
%% Usted es libre de usarlo, modificarlo o distribuirlo siempre que se
%% respeten los términos de la licencia y se reconozca al autor original
%% -------------------------------------------------------------------------
%% Estos comandos permiten dibujar rápidamente un intervalo de cualquier
%% tipo en recta real.
%% -------------------------------------------------------------------------
%% Paquetes necesarios
\usepackage{tikz}
\usetikzlibrary{babel, shapes}
\usepackage{amsmath}
\usepackage{amsfonts}
\usepackage{amssymb}
\usepackage{xcolor}
%% -------------------------------------------------------------------------

%----------------------------------------------------
% Parámetros de entrada
% #1 cota inferior
% #2 cota superior
% #3 posición vertical del intervalo
% #4 fill o fill = none o <-, tipo de cota
% #5 fill o fill = none o ->, tipo de cota
% #6 nonInf o inf, tipo de intervalo
% #7 color
\newcommand{\interval}[7]{
    \IfEqCase {#6}{
        {nonInf}{
        \node [circle, draw, #4, line width = 1.5pt, color = #7, inner sep = 0pt, minimum size = 5pt] (ci) at (#1, #3) {};
        \node [circle, draw, #5, line width = 1.5pt, color = #7, inner sep = 0pt, minimum size = 5pt] (cs) at (#2, #3) {};
        \draw [line width = 1.5pt, color = #7] (ci) -- (cs);
        \draw (ci) -- (#1, -.2);
        \node (tag) at (#1, -.4) {#1};
        \draw (cs) -- (#2, -.2);
        \node (tag) at (#2, -.4) {#2};
        }
        {inf}{
        	\IfEqCase {#4}{
            	{<-}{
			        \node [circle, draw, #5, line width = 1.5pt, color = #7, inner sep = 0pt, minimum size = 5pt] (cs) at (#2, #3) {};
                    \draw [#4, line width = 1.5pt, color = #7] (#1 - 1.5, #3) -- (cs);
                    \draw (cs) -- (#2, -.2);
                    \node (tag) at (#2, -.4) {#2};
                }
                {->}{
                	\node [circle, draw, #5, line width = 1.5pt, color = #7, inner sep = 0pt, minimum size = 5pt] (ci) at (#1, #3) {};
                    \draw [->, line width = 1.5pt, color = #7] (ci) -- (#2 + 1.5, #3);
                    \draw (ci) -- (#1, -.2);
                    \node (tag) at (#1, -.4) {#1};
                }
            }
        }
        }[\PackageError{tree}{Undefined option to intervals: #6}{}]
}
%----------------------------------------------------
% Recta Real
\newcommand{\reaLine}[2]{
    \node (li) at (#1 - 1.5, 0) {};
    \node (ls) at (#2 + 1.5, 0) {$\mathbb{R}$};
	\draw [<->] (li) -- (ls);
}
\usepackage{chemfig}

\newcommand{\cafeina}[1]{
	\scalebox{#1}{
		\chemfig{*6((=O)-N(-CH_3)-*5(-N=-N(-CH_3)-=)--(=O)-N(-H_3C)-)}
	}
}

\newcommand{\benceno}[1]{
	\scalebox{#1}{
		\chemfig{*6(=-=-=-)}
	}
}

\newcommand{\adrenalina}[1]{
	\scalebox{#1}{
		\definesubmol{&}{-[,,,,draw=none]}
		\definesubmol{&&}{-[,,,2,draw=none]}
		\chemfig{*6((-HO)-=*6(!&!{&&}HN(-CH_3)-[,,2]-(<0H)-)-=-(-HO)=)}
	}
}
\usepackage{tcolorbox}

\newtheorem{definition}{Definición}

\newtheorem{ejemplo}{Ejemplo}[section]

\newenvironment{remark}[1]
{
	\begin{tcolorbox}[colback = myBlue!25, colframe = blue254!75, title=#1, arc = 3mm, sharp corners = northwest]
	\fontfamily{qag}\selectfont
}
{
	\end{tcolorbox}
}

\DeclareRobustCommand{\eqCuadratica}{
	\begin{equation}
		x_{1,2} = \frac{1}{2a}\Bigl(-b \pm\sqrt{b^2 - 4ac}\Bigr)
	\end{equation}
}

\begin{document}
\maketitle
En este ejemplo vamos a explorar la definición de comandos cada vez más elaborados y también vamos a ver cómo definir nuevos ambientes. Definir comandos y ambientes es una tarea que nos puede ahorrar mucho tiempo y trabajo, veamos algunos ejemplos.

\section{Definición y uso de \texttt{newcommand}}

Aquí voy a escribir la ecuación cuadrática
\eqCuadratica{}

Ahora, a partir del diccionario de ecuaciones \texttt{dict\_ecuaiones.tex} voy a escribir las ecuaciones de Maxwell

\textcolor{blue254}{\Maxwell{}}

La transformada de Fourier puede presentarse de tres formas diferentes, en línea \Fourier[inLine], como una ecuación numerada

\Fourier{}

o en formado display

\Fourier[disp]

y ¿qué tal algo más elaborado como la molécula de la cafeína?
\begin{center}
	\cafeina{.5}
\end{center}

o la del benceno
\begin{center}
	\benceno{.5}
\end{center}

y ¿qué tal la de la adrenalina?
\begin{center}
	\adrenalina{.5}
\end{center}

Hagamos algo más complejo, que tal comandos con parámetros de entrada condicionados, los comandos \texttt{reaLine} (con dos parámetros de entrada) e \texttt{interval} (con siete parámetros de entrada) del archivo \texttt{intervalos.tex} permiten construir rápidamente la representación gráfica de un intervalo sobre la recta real.

\begin{figure}[ht]
	\centering
	\begin{subfigure}[t]{.47\textwidth}
	\centering
		\begin{tikzpicture}
			\reaLine{0}{2}
			\interval{0}{2}{.3}{<-}{fill = none}{inf}{myBlue}
		\end{tikzpicture}
	\caption{Intervalo $(-\infty, 2)$}
	\end{subfigure}
	\hfill
	\begin{subfigure}[t]{.47\textwidth}
	\centering
		\begin{tikzpicture}
			\reaLine{2}{4}
			\interval{2}{4}{.3}{->}{fill = none}{inf}{myBlue}
		\end{tikzpicture}
	\caption{Intervalo $(2, \infty)$}
	\end{subfigure}
	\hfill
	\begin{subfigure}[t]{1\textwidth}
	\centering
		\begin{tikzpicture}
			\reaLine{-2}{2}
			\interval{-2}{2}{.3}{fill}{fill = none}{nonInf}{myBlue}
		\end{tikzpicture}
	\caption{Intervalo $[-2,2)$}
	\end{subfigure}	
	\caption{Uso del comando \texttt{interval}}
\end{figure}

\section{Definición y uso de \texttt{newenvironment}}

Ahora trabajemos con la definición de ambientes, para empezar un ambiente numerado con \texttt{newtheorem}

\begin{definition}
Se dice que un par de funciones $f_1$ y $f_2$ son \textbf{linealmente dependientes} en un intervalo $I$ si existe un par de constantes $c_1$ y $c_2$ no ambas nulas de tal forma que
\[
	c_1f_1(x) + c_2f_2(x) = 0, \;\;\;\; \forall x \in I,
\]
si tales contantes no existen, las funciones se dicen \textbf{linealmente independientes}, es decir $f_1$ y $f_2$ son \textbf{linealmente independientes} si y solo si $c_1f_1(x) + x_2f_2(x) = 0$ para $c_1 = c_2 = 0$ .
\end{definition}

También se pueden crear ambientes numerados para cada sección del documento

\begin{ejemplo}
	Dado el problema de valor inicial
	\[
		9y'' + 6y' + 82y + 0, \;\;\; y_0(0) = 1 \text{ e } y_0'(0) = 2
	\]
	al sustituir $y = e^{\lambda x}$ se obtiene la ecuación característica
	\[
		9\lambda^2 + 6\lambda + 82 = 0
	\]
	cuyas raíces son los números complejos conjugados
	\[
		\lambda_1 = -\frac{1}{3} - 3i \text{ y } \lambda_2 = -\frac{1}{3} + 3i,
	\]
	....
\end{ejemplo}

\begin{remark}{Importante}
	Puedes utilizar comandos y ambientes uno sobre otros, por ejemplo, vamos a utilizar el comando \texttt{interval} para definir en el ambiente \texttt{remark}, un intervalo...
	\begin{center}
		\begin{tikzpicture}
			\reaLine{-1}{3}
			\interval{-1}{3}{.3}{fill}{fill}{nonInf}{orange22}
		\end{tikzpicture}
	\end{center}
	también se puede incluir alguna de nuestras moléculas
	\begin{center}
		\cafeina{.6}
	\end{center}
\end{remark}

\end{document}